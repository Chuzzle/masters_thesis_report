\documentclass[10pt,a4paper]{article}
\usepackage[utf8]{inputenc}
\usepackage{amsmath}
\usepackage{amsfonts}
\usepackage{amssymb}
\usepackage{graphicx}
\usepackage{bm}
\usepackage{wrapfig}
\usepackage{appendix}
\usepackage{subcaption}
\usepackage{mdframed}

\usepackage{listings}
\usepackage{color}
\lstdefinestyle{matlab_custom}{
	language = matlab,
	basicstyle=\scriptsize\fontfamily{cm}\sffamily,
	showstringspaces = false,
}

\usepackage[margin=2cm]{geometry} %Set margin size

\usepackage{fancyhdr}

\pagestyle{fancy}
\fancyhf{}
\rhead{Hanna Autio}
\lhead{\today}
\chead{Masters Thesis} %Course name

\cfoot{\thepage}

\usepackage{cleveref}

\author{Hanna Autio}
\title{A stochastic model for Meningococcal Disease in the African Meningitis belt} %Title here
\date{\today}

\begin{document}
\maketitle
\thispagestyle{empty}
\cleardoublepage
\newpage
\setcounter{page}{1}

% On my indices:
% 	- Greek letters are constants
% 	- n \eta for populations
% 	- k \kappa for events


\section{Introduction}

The mathematical description of populations and disease dynamics is a field which can be extremely complex. It is a subject of much research, relevant to policy makers and medical researchers.


% Disease modelling
%		-stochastic v deterministic
%		-incorporating weather and shit


% The förutsättningar for different countries to deal with the problem
%		-When is an epidemic?
%		-Look into more stuff to put here.
% Issues with modelling weather + how that makes it hard to say exactly what's going on with the seasonality
% Something about meningitis + WHO
% 	-vaccination
%		-possible eradication
%		-areas of research which they have identified

% MenAfriCar?

\subsection{Background}

The bacterium responsible for the disease examined in this report is the \emph{Neisseria Meningitidis}. Humans are the only known reservoir for the bacterium, and at a given time somewhere between $2\% - 50\%$ of a population are likely to be carriers of the bacterium [sources]. In general, carriage does not lead to invasive disease, but it has been linked to a subsequent immune response to the bacterium. Carriage can last both for a very short period of time as well as for several months, during which the bacterium is present in the nasopharynx of the carrier \cite{taha2002duality}. 

\emph{N. Meningitidis} is genetically variable. It is classified, based on surface-level structures, into different serogroups, most commonly A, B, C, W-135, X, and Y [source]. Note however that it is fairly common for the bacterium to be non-serogroupable[source?]. Of the main serotypes, most epidemics have historically been caused by serogroups A and B, and serogroups C and W-135 have also been the cause of some epidemics. 

The African meningitis belt was first described by Lapeyssonnie in 1968 \cite{lapeyssonnie}. This region is shown in \cref{[FIGURE:MENINGITIS BELT]} and has been characterized by semi-regular epidemics documented since the early 20th century. Its specific geographical area is defined be the WHO as the region in Africa between Senegal in the west and Ethiopia in the east [source], which coincides fairly well with the area described by Lapeyssonnie.

Meningitis dynamics in the region have some characteristics. Firstly, there are recurring epidemics with intervals of about 5-15 years. Secondly, the incidence rates of disease is higher than in other regions of the world. Thirdly, the disease is most commonly caused by a serogroup which does not usually cause epidemics in the rest of the world. All of these properties are also affected by a strong seasonality, so that there are three distinguishable states for the disease in a population. There is the hypoendemic state, with a low incidence rate of about [] which occurs during the wet season. During the dry season, the system falls into a hyperendemic state with increased incidence rate, or occasionally an epidemic state. An epidemic state is defined as a rate of disease above []. \cite{mueller2010hypothetical}

The reason for the specific dynamics in this region is often considered to be some climatic factors. One such factor frequently considered is the total precipitation. The region coincides fairly well with the area between the 300mm and 1100mm isohyets \cite{molesworth2002meningitis}. Another common cilmate factor for the area, as defined by the WHO, is that it is affected by the Harmattanm a dry wind from the Saharan desert. As there are links between factors that irritate the airways and subsequent infection by \emph{N. Meningitidis} [sources on smoking, colds], a link between the Harmattan and the hyperendemic meningitis season would seem plausible. Is is an area of active research [?], and finding any climactic factors that could be driving the epidemics is outlined as an area of priority by the WHO [source: research priorities].

Historically, epidemics in the meningitis belt have been caused by serogroup A, while most epidemics and infections in the rest of the world has been casued by serogroup B [source]. To tackle the disease dynamics in the meningitis belt, vaccines towards serogroup A have been introduced, and since the introduction of these, no new epidemics caused by this serogroup has occurred in the area. However, there have been incidences of other serogroups causing epidemics, most notably W-135. [sources]

% Describe the disease itself, symptoms and also overarching dynamics. Clonal waves.

% Describe the seasonality
% How the disease spreads
% Serotypes + different strains of invasiveness + mutations
% Age distribution and immunity and how it relates
% Carriage state + immunization
% Patterns in other parts of the world and how it compares (most commonly infects children who loses immunity from the mother) + carriage in college students and military recruits

% Social aspects in the meningitis belt
% What are the goals of this thesis and why have I chosen to focus on those aspects?

\subsection{Terminology}
% Event
% System
% probability rate
% population/subpopulation

\section{Methods}

The goal of this project is to develop methods to understand the dynamics characterizing meningococcal diseases in the meningitis belt. The degree of urbanization is comparatively low in this region [source] and so any applicable method must be accurate also for smaller communities. Traditional methods using deterministic dynamics and non-discrete populations become gradually more imprecise when dealing with decreasing population sizes [source], and as such a different approach is required.

Hence, we develop a dynamical model for integer-valued populations following a Markov-Jump process. The model is subsequently simulated using the Feller-Kendall algorithm, and it's behaviour is validated by comparing to real-world data as well as previous models (on an appropriate scale).

% Make a comparison to earlier modelling examples and comment on why they may be less accurate

\subsection{Model}

% Population into subpopulations
% System state?
% Non-unique indivduals
% Event probabilites do not change between events
% Interaction time <-> population

\subsubsection{Populations}

\subsubsection{Time dependency}

\subsection{Implementation}

\section{Results and analysis}

\section{Conclusions}


% Figure float, including 2 subfigures
%\begin{figure}
%		\begin{mdframed}
%			\centering
%			\begin{subfigure}{0.45\textwidth}
%				\begin{mdframed}
%					\includegraphics[width=\textwidth]{}
%					\caption{} \label{}
%				\end{mdframed}
%			\end{subfigure}
%			\begin{subfigure}{0.45\textwidth}
%				\begin{mdframed}
%					\includegraphics[width=\textwidth]{}
%					\caption{} \label{}
%				\end{mdframed}
%			\end{subfigure}
%			\caption{} \label{}
%		\end{mdframed}
%\end{figure}

%\begin{wrapfigure}{r}{0.5\textwidth}
%		\begin{mdframed}
%			\includegraphics[width=0.48\textwidth]{}
%			\caption{} \label{}
%		\end{mdframed}
%\end{wrapfigure}

\newpage

\begin{appendices}

%\lstinputlisting[language=matlab, style=matlab_custom]{} % Example code input

\end{appendices}

\bibliographystyle{plain}
\bibliography{bibliography}


\end{document}